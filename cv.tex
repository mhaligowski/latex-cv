\documentclass[1pt,a4paper]{moderncv}

% moderncv themes
\moderncvtheme[blue]{classic}                 % optional argument are 'blue' (default), 'orange', 'red', 'green', 'grey' and 'roman' (for roman fonts, instead of sans serif fonts)

% character encoding
\usepackage[utf8]{inputenc}                   % replace by the encoding you are using

% adjust the page margins
\usepackage[scale=0.8]{geometry}
%\setlength{\hintscolumnwidth}{3cm}						% if you want to change the width of the column with the dates
%\AtBeginDocument{\setlength{\maketitlenamewidth}{6cm}}  % only for the classic theme, if you want to change the width of your name placeholder (to leave more space for your address details
%\AtBeginDocument{\recomputelengths}                     % required when changes are made to page layout lengths

% personal data
\firstname{Mateusz}
\familyname{Haligowski}
\title{Curriculum Vitae}               % optional, remove the line if not wanted
\address{ul. Stawna 1C/45}{81-629 Gdynia}    % optional, remove the line if not wanted
\mobile{+48\,503\,702\,923}                    % optional, remove the line if not wanted
\email{mhaligowski@gmail.com}                      % optional, remove the line if not wanted
\extrainfo{http://github.com/mhaligowski}

% to show numerical labels in the biblaphy; only useful if you make citations in your resume
\makeatletter
\renewcommand*{\bibliographyitemlabel}{\@biblabel{\arabic{enumiv}}}
\makeatother

% bibliography with mutiple entries
%\usepackage{multibib}
%\newcites{book,misc}{{Books},{Others}}

%\nopagenumbers{}                             % uncomment to suppress automatic page numbering for CVs longer than one page
%----------------------------------------------------------------------------------
%            content
%----------------------------------------------------------------------------------
\begin{document}
\maketitle

\section{Doświadczenie}
\subsection{Zawodowe}
\cventry{XI 2012 -- }{Senior Developer}{JIT Solutions sp. z o. o./Nordea Bank Polska (konsultant)}{Gdynia}{}{Pracuję jako konsultant w jednym z większych banków detalicznych w Polsce. Implementuję rozwiązania na potrzeby procesów wewnętrznych z wykorzystaniem technologii firmy Oracle, takich jak PL/SQL i Oracle BPM.
}
\cventry{X~2011 -- XI 2012}{Programista JEE/Ruby on Rails}{SpeedNet sp. z o. o.}{Gdynia}{}{Brałem udział w projekcie o średnim rozmiarze z branży ubezpieczeniowej. Wykonywałem bieżące zadania zarówno dotyczące warstwy klienta (Google Web Toolkit, MVP4g, Google Gin), jak i po stronie serwera (IBM WebSphere AS, JBoss AS, PostgreSQL, Oracle 10g, Google Guice, Hibernate, Drools). Byłem odpowiedzialny m.in. za migrację aplikacji ze środowiska JBoss AS/PostgreSQL na środowisko IBM WebSphere AS/Oracle 10g.
Ponadto, na potrzeby wewnętrzne wykonałem wtyczkę do systemu zarządzania projektami Redmine w technologii Ruby on Rails.
}

\cventry{I~2011 -- IX~2011}{Programista JEE}{Lufthansa Systems Poland/Hamburg S\"{u}d}{Gdańsk/Hamburg}{}{Jako programista JEE zatrudniony przez firmę Lufthansa Systems Poland wykonywałem pracę w Hamburgu, w firmie Hamburg S\"{u}d, jednym z największych przedsiębiorstw zajmujących się transportem morskim. Projekt polegał na wytworzeniu aplikacji do zarządzania flotą statków, a w jego ramach zajmowałem się implementacją przekazywania wiadomości między własną bazą danych a klientem zewnętrznym z wykorzystaniem technologii EJB oraz JMS na serwerze aplikacji OC4j i bazie danych Oracle 11g. Ponadto byłem odpowiedzialny za wdrożenie aplikacji Arquillian, służącej do wykonywania testów jednostkowych w kontenerach JEE.}

\cventry{II~2010 -- XII~2010}{Działalność gospodarcza}{Nostromo Labs}{Gdynia}{}{Próbowałem własnych sił jako przedsiębiorca, oferując usługi z dziedziny wytwarzania oprogramowania. Koncentrowałem się na aplikacjach typu  Internet Application z wykorzystaniem technologii Python (Django), Java (GWT, Java Servlets), Ruby (Rails). Działalność zdecydowałem się zamknąć ze względu na brak nowych zleceń.}

\cventry{IX~2009 -- I~2010}{Programista JEE}{sprezentuj.pl}{Praca zdalna}{}{Pracowałem zdalnie jako programista Java w start-upie internetowym sprezentuj.pl, będącym internetową listą prezentów. Cechą charakterystyczną aplikacji był okienkowy interfejs użytkownika, wykonany za pomocą biblioteki GWT. Ponadto nauczyłem się również technologii Spring, Google Guice oraz Hibernate, wykorzystywanych w niższych warstwach oraz praktycznego zastosowania wzorców projektowych.}

\cventry{VI~2008 -- IX~2009}{Specjalista ds. Zarządzania Ryzykiem Kredytów Detalicznych}{Meritum Bank ICB S.A.}{Gdańsk}{}{W ramach pracy w banku zajmowałem się tworzeniem i rozwojem narzędzi wspierających system zarządzania ryzykiem kredytów detalicznych w banku, oceniającym zdolność kredytową klientów oraz komunikacją z podmiotami zewnętrznymi.Wśród wykonanych narzędzi należy wymienić kalkulator oceniający klientów (wykonany w języku programowania Python i języku zapytań SQL), system raportujący dane o klientach do Biura Informacji Kredytowej (wykonany w Javie, JDBC i języku SQL) oraz systemy informujące o poszczególnych wskaźnikach (w języku SQL).}

\cventry{VIII~2007 -- X~2007}{Praktykant}{Thomson Reuters}{Gdynia}{}{W ramach praktyk studenckich zajmowałem się weryfikacją pobranych danych makroekonomicznych.}

\subsection{Kursy, szkolenia i certyfikaty}
\cventry{XII~2010}{Szkolenie Adobe InDesign}{Poziom podstawowy}{Combidata}{Gdynia}{}{}
\cventry{IX~2010}{Szkolenie Adobe Illustrator}{Poziom podstawowy}{Combidata}{Gdynia}{}{}
\cventry{VII~2010}{Projektowanie graficzne w reklamie i publikacjach}{}{Combidata}{Gdynia}{}
\cventry{VII~2007}{Oracle PL/SQL z elementami strojenia dla analityków danych}{}{Altkom Akademia}{Gdynia}{}
\cventry{X~2006}{Sun Certified Java Programmer for Java 5}{}{}{}{}

\subsection{Pozostałe doświadczenie}
\cventry{XI 2012}{prowadzący}{3hack.pl}{}{}{Warsztaty Apache Hadoop.}{}
\cventry{2012 --}{prelegent}{Trójmiasto Java User Group}{}{}{Prelekcje dotyczące narzędzi Hadoop oraz Vagrant.}{}
\cventry{X 2010}{prelegent}{PyConPL'2010}{}{}{Real-time internet applications in Python}{}
\cventry{V~2008}{prelegent}{Konferencja ,,Metody numeryczne w naukach społecznych''}{Koło~Naukowe Metod Ilościowych na Wydziale~Zarządzania UG}{}{Python w zastosowaniach naukowych}{}
\cventry{IV 2008}{prelegent}{RuPy'2008}{}{}{Scientific Applications of Python}{}
\cventry{V~2007 -- VIII~2007}{uczestnik-student}{Google Summer Of Code 2007}{Scribus Team}{}{Brałem udział w rozwoju oprogramowania Open Source do elektronicznego składu publikacji Scribus, w ramach programu stypendialnego firmy Google. Projekt wtyczki służącej do impozycji wykonałem w języku C++, z wykorzystaniem biblioteki Qt.}{}

\section{Wykształcenie}
\cventry{X~2008 -- IX~2010}{studia magisterskie}{Politechnika Gdańska}{Wydział Elektroniki, Telekomunikacji i Informatyki}{kier. Informatyka}{Tytuł pracy magisterskiej: Szeregowanie zadań przy niepewnych parametrach, \tiny{\emph{w trakcie pisania pracy}}}  % arguments 3 to 6 can be left empty
\cventry{X~2008 -- IX~2010}{studia magisterskie}{Uniwersytet Gdański}{Wydział Zarządzania}{kier. Informatyka i Ekonometria, spec. Ekonometria i Statystyka}{Tytuł pracy magisterskiej: Analiza efektywności funduszy emerytalnych w latach 2002-2010, \tiny{przerwane}}  % arguments 3 to 6 can be left empty
\cventry{X~2005 -- IX~2008}{studia licencjackie}{Uniwersytet Gdański}{Wydział Zarządzania}{kier. Informatyka i Ekonometria, spec. Ekonometria i Statystyka}{Tytuł pracy licencjackiej: Analiza efektywności spółek giełdowych na wielu giełdach (praca w jęz. angielskim)}  % arguments 3 to 6 can be left empty
\cventry{X~2005 -- VI~2006}{studium policealne}{English Unlimited}{Studium Tłumaczy Języka Angielskiego}{}{}{}
\cventry{IX~2002 -- V~2005}{liceum ogólnokształcące}{III LO im. Marynarki Wojennej RP w Gdyni}{klasa o profilu matematyczno-informatycznym}{}{}{}

\section{Języki obce}
\cvlanguage{angielski}{biegły}{Certificate of Proficiency in English (C1), Test of English as a Foreign Language (99/120 pkt.)}
\cvlanguage{niemiecki}{samodzielny}{Zertifikat Deutsch (B1)}

\newpage{}

\section{Umiejętności}
\cvcomputer{Języki programowania}{Java, Python, Javascript, C/C++, PL/SQL}{Narzędzia analityczne}{R, MatLab, Statistica, SPSS, SQL}
\cvcomputer{Narzędzia graficzne}{Pakiet Scribus, Gimp, \LaTeX}{Środowiska programistyczne}{Eclipse, Emacs, vim, XCode}
\cvcomputer{Kontenery JEE}{IBM WebSphere AS, JBoss AS, Apache Tomcat, OC4J, Glassfish, OpenEJB}{}{}

\section{Zainteresowania}
\cvline{Muzyka}{\small Gram na gitarze basowej w zespole bluesowym Turbocola.}
\cvline{Książki}{\small Głównie fantasy i science-fiction, jednak chętnie sięgam również po inne gatunki.}
\cvline{Elektronika}{\small W wolnym czasie chętnie zajmuję się elektroniką, głównie technologią mikroprocesorową z wykorzystaniem platformy Arduino.}
\cvline{Nauka i nowe technologie}{\small Pasjonuję się każdym rodzajem nowych technologii i z chęcią czytam artykuły naukowe i książki popularno-naukowe.}
\cvline{Analiza danych}{\small Jako absolwent kierunku ekonometria interesuję się również przetwarzaniem danych i ich analizą. } 

\end{document}


%% end of file `template_en.tex'.
